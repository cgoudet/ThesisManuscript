%===============================================
\section{Measurement of Higgs boson couplings}
\frame{\tableofcontents[currentsection]}


\begin{frame}{$\mu_{\gamma\gamma}$ measurement}
$\mu_{\gamma\gamma}$ is a main variable to measure. It is related to the cross section (production probability) :
$$\mu_{\gamma\gamma}=\frac{(\sigma\times BR)^{meas}}{(\sigma\times BR)^{SM}}=1.17 \pm 0.23 \text{(stat)}\ ^{+0.10}_{-0.08}\text{(syst)}\ ^{+0.12}_{-0.08} \text{(theory)}$$
  \begin{minipage}{0.49\linewidth}
    \includegraphics[width=\linewidth]{1408_7084_19f.pdf}
  \end{minipage}
  \hfill
  \begin{minipage}{0.49\linewidth}
    \begin{tikzpicture}
      \node[anchor=south west] { \includegraphics[width=\linewidth]{1408_7084_19t.pdf} };
%      \draw[step=1.0,black,thin] (0,0) grid (10,4);
      \draw[red, line width=1mm, rounded corners =2pt] ( 1, 2.05 ) rectangle ( 13.7, 3.95 ) ;
    \end{tikzpicture}
  \end{minipage}
If no improvements, \textcolor{blue}{\bf calibration uncertainty will be dominant in run 2}.
\end{frame}

\begin{frame}{Likelihood Method}
A function ({\bf likelihood}) is built to {\bf evaluate the best set of parameters ($\vec{\mu}$,$\vec{\theta}$)} of a model to agree the best with a dataset in a category.

$$\mathcal{L}=\underbrace{\frac{(n_{s}(\vec{\mu},\vec{\theta})+b)^{n_{obs}}}{n_{obs}!} e^{-(n_{s}(\vec{\mu},\vec{\theta})+b)}}_{\textcolor{red}{\text{(1)}}}  \overbrace{\prod_j^{n_{obs}}\psi(\vec{x_j};\vec{\mu},\vec{\theta})}^{\textcolor{violet}{\text{(2)}}} \underbrace{e^{-\frac{\theta^2}{2}}}_{\textcolor{blue}{\text{(3)}}}$$
\vfill
\begin{minipage}{0.49\linewidth}
\textcolor{red}{(1) {\bf Poissonian law} to evaluate the probability to observe $n_{obs}$($\equiv$ signal $+$ background) events when $(n_s+b)$ are expected.}\newline
\textcolor{violet}{(2) {\bf Probability density function} of the observables $\vec{x}$ (diphoton invariant mass for example) for the $j^{th}$ event.}\newline
\textcolor{blue}{(3) Constraint on the nuisance parameter $\theta$. See next slide.}\newline
\end{minipage}
\begin{minipage}{0.49\linewidth}
\includegraphics[width=\linewidth]{Cgam_009.png}
\end{minipage}
\end{frame}




\begin{frame}{Nuisance parameters}
There are some {\bf external measurements}  that contribute to the likelihood and have some {\bf uncertainties}. 
A {\bf free nuisance parameter} is added for each of these measurements.
In order  to take into account these external measurements, a {\bf constraint is put on these nuisance parameters}. 
\vfill
For example, the luminosity is re-defined as  $L(1+\delta_L \theta_L)$, with $\theta_L$ the nuisance parameter and $\delta_L$ the uncertainty on the luminosity (assumed to be Gaussian).
In this case, a Gaussian constraint is chosen.

The contribution from luminosity will hence be :
$$L(1+\delta_L \theta_L)e^{-\theta_L^2/2}$$
\vfill
\textcolor{blue}{\bf Error Estimation}\newline
A test statistic is defined as : $t_\mu=-2ln\frac{\mathcal{L}(\mu,\hat{\hat{\theta}})}{\mathcal{L}(\hat{\mu},\hat{\theta})}$, with $\hat{\theta}$ and $\hat{\mu}$ the best (fitted) parameters, and $\hat{\hat{\theta}}$ the fitted nuisance parameters for a fixed $\mu$.\newline
Uncertainty are given by : \textcolor{red}{$\mathbf{t_{\hat{\mu}\pm 1\sigma}=1}$} and \textcolor{red}{$\mathbf{t_{\hat{\mu}\pm 2\sigma}=4}$} in 1D Gaussian limit.
\end{frame}

