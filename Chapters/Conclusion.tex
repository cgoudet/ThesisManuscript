\chapter{Conclusion}

The run 1 of the LHC was a major milestone in high energy physics.
The outstanding performances of the accelerator allowed the experiments to collect a large set of high quality data.
A highlight of these data was the discovery of the Higgs boson and the measurement of its properties, in particular its mass (last free parameter of the Standard Model) and its couplings.
After two years of shutdown for consolidation and improvement of the accelerator and the detectors, the run 2 started in 2015 at a higher energy ($\sqrt{s}=13$ TeV) and with a higher luminosity, with the hope of discovering hints of Beyond the Standard Model physics, and the insurance to improve the precision on the Higgs boson parameters.

This thesis describes my work in the improvement of precision in the measurement of the Higgs boson couplings.
The first part focuses on the measurement of the $Z\rightarrow ee$ in-situ energy scale factors and resolution constant term of the electromagnetic calorimeter, used to correct the energy of electrons and photons, and which was a dominant experimental uncertainty on the H boson cross section in the diphoton canal at run 1.
Pre-recommendations for the beginning of the run 2 have been computed.
Then, several calibration models have been provided for various public results.
The improvement of statistics and methodology allowed to reduce by a factor two the systematic uncertainty on the resolution constant term for the run 2 couplings measurement.

The measurements of the H boson couplings and mass have been performed with $36$ fb$^{-1}$ of data collected in 2015 and 2016, using this new calibration.
Thanks to a lot of improvements on several sides, the uncertainty of the inclusive signal strength has been reduced by a factor two.
The observed global signal strength is $\mu=0.99 \pm 0.14$ in agreement with the Standard Model.
The H boson mass has also been measured at $m_H=125.11\pm 0.42$ GeV.