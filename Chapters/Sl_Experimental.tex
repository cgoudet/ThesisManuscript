\section{Experimental conditions and data processing}
\frame{\tableofcontents[currentsection]}

\begin{frame}{The Large Hadron Collider}
  The LHC aims at accelerating and colliding protons.
  Analysing debris of collisions allows to probe SM and/or beyond.
  \vfill
  
  \begin{minipage}{0.49\linewidth}
    \begin{itemize}
    \item located at Geneva.
    \item 27 km long.
    \item 100 m underground
    \item collision every 25 ns.
    \item $\sqrt{s}=13$ TeV
    \item 4 collision points equipped with detectors : ALICE, ATLAS, CMS and LHCb.
    \end{itemize}
    \end{minipage}
  \hfill
  \begin{minipage}{0.49\linewidth}
    \includegraphics[width=\linewidth]{LHC.jpg}
    \end{minipage}
\end{frame}
%%===========================
\begin{frame}{LHC data taking condition}
  The collision conditions at LHC have significantly changed since its contruction.
  \begin{itemize}
  \item Major increase of integrated luminosity per year.
  \item Large increase in colliisions per bunch crossing.
  \end{itemize}

  \begin{center} \includegraphics[width=0.6\linewidth]{ATLASLumiYear.pdf} \end{center}
\end{frame}
%%===========================
\begin{frame}{General purpose apparatus}
  A general purpose particle detector (ATLAS and CMS) is designed as layers with specific measurement goal.
  
  \begin{center}    \includegraphics[width=0.7\linewidth]{detector.jpg} \end{center}
\end{frame}

%%===========================
\begin{frame}{ATLAS experiment}
  \begin{minipage}{0.49\linewidth}
    \includegraphics[width=\linewidth]{ATLASExperiment_1f1.pdf}\\
    Performance goals of the ATLAS detector\\
    \includegraphics[width=\linewidth]{ATLASExperiment_1t1.pdf}
  \end{minipage}
  \hfill
  \begin{minipage}{0.49\linewidth}
    \begin{itemize}
    \item {\bf Large acceptance}
    \item {\bf Radiation hard}
      \newline
    \item \textcolor{blue}{\bf Silicon and TRT tracker in 2T magnetic field}\\
      Measure position and momentum of charged particles
    \item \textcolor{blue}{\bf Liquid argon electromagnetic calorimeter (LAr)} \\
      Measure energy of electrons and photons.
    \item {\bf Scintillating tiles hadronic calorimeter} \\
      Measure energy of jets
    \item {\bf Muon chambers}
    \end{itemize}
  \end{minipage}\\
\end{frame}

%%==========================
\begin{frame}{ Electromagnetic calorimeter (LAr) }
  \begin{minipage}{0.4\linewidth}
    \includegraphics[width=\linewidth]{MarcHDR_2f30.pdf}\newline
    \centering
    \includegraphics[width=\linewidth]{ATLASExperiment_5f4.pdf}
  \end{minipage}
  \hfill
  \begin{minipage}{0.59\linewidth}
    \begin{itemize}
    \item $1.4$m $<r<2$m
    \item Sampling calorimeter : \\
      - absorber : lead\\
      - active material : \textcolor{blue}{\bf Liquid Argon }($88$K)\\
    \item {\bf Accordion geometry} gives uniformity and hermeticity along $\phi$.
    \item {\bf Longitudinally segmented} for pion discrimination
      
    \end{itemize}
    \centering
    \includegraphics[width=0.6\linewidth]{shower.jpg}
  \end{minipage}
\end{frame}

%=============================
\begin{frame}{Data recording}
  \begin{itemize}
    \item Particle passing through LAr $\rightarrow$ electric signal in electrode.
    \item Need to collect signal information with limited data size
  \end{itemize}

  \begin{minipage}{0.49\linewidth}
    \begin{itemize}
    \item Bi-polar filter removes pile-up
    \item 4 samples (1 per 25ns)
    \item $A = \sum_{j=1}^{N_{sample}} a_j(s_j-p)$

\vfill
    \item $E = F_{\mu \mbox{A}\rightarrow \mbox{MeV}}  $ \\ $ \times    F_{\mbox{DAC}\rightarrow\mu\mbox{A}}$  \\  $  \times     \frac{1}{\frac{M_{\mbox{phys}}}{M_{\mbox{cali}}}} $  \\  $\times G \times A $
      \end{itemize}
  \end{minipage}
  \hfill
  \begin{minipage}{0.49\linewidth}
    \includegraphics[width=\linewidth]{jinst10_09_p09003_f3.pdf}
  \end{minipage}

\end{frame}
%=============================
\begin{frame}{EM object reconstruction}
  Need to transform the signal in a set of cells into a particle object.
\vfill
  \begin{minipage}[t]{0.49\linewidth}
    {\bf Cluster reconstruction }
    \begin{itemize}
    \item Sliding window algorithm
    \item Search deposit $>2.5$~GeV in $3\times 5$ L2 cells towers.
    \item Energy cluster defined as $3\times 7$.
    \end{itemize}
    \end{minipage}
    \hfill
    \begin{minipage}[t]{0.49\linewidth}
      {\bf Track reconstructio }
      \begin{itemize}
      \item Search 3 compatible points in Si and propagate track to TRT
      \item Search 3 compatible points in TRT and propagate track to Si
      \item Identify conversion vertices
      \end{itemize}
    \end{minipage}
    \vfill
  If a track match a cluster $\rightarrow$ electron
  else if conversion tracks match clusters $\rightarrow$ converted photon
  else $\rightarrow$ unconverted photon.
\end{frame}
