\section{Experimental conditions and data processing}
\frame{\tableofcontents[currentsection]}

\begin{frame}{The Large Hadron Collider}
  The LHC aims at accelerating and colliding protons.
  Analysing products of collisions allows to probe SM and/or beyond.
  \vfill
  
  \begin{minipage}{0.49\linewidth}
    \begin{itemize}
    \item located at Geneva.
    \item 27 km long.
    \item 100 m underground
    \item collision every 25 ns.
    \item nominal luminosity of $10^{34}$~cm$^{-2}$s$^{-1}$
    \item $\sqrt{s}=13$ TeV
    \item 4 collision points equipped with detectors : ALICE, ATLAS, CMS and LHCb.
    \end{itemize}
    \end{minipage}
  \hfill
  \begin{minipage}{0.49\linewidth}
    \includegraphics[width=\linewidth]{LHC.jpg}
    \end{minipage}
\end{frame}
%%===========================
\begin{frame}{LHC data taking condition}
  The collision conditions at LHC have significantly changed since its contruction.
  \begin{itemize}
  \item Major increase of integrated luminosity per year.
  \item Large increase in collisions per bunch crossing.
  \end{itemize}

  \begin{center} \includegraphics[width=0.49\linewidth]{ATLASLumiYear.pdf}
  \includegraphics[width=0.49\linewidth]{mu_2015_2016} \end{center}
\end{frame}
%%===========================
\begin{frame}{ATLAS experiment}
  \begin{minipage}{0.49\linewidth}
    \includegraphics[width=\linewidth]{ATLASExperiment_1f1.pdf}\\
    \includegraphics[width=\linewidth]{detector.jpg}
  \end{minipage}
  \hfill
  \begin{minipage}{0.49\linewidth}
    \begin{itemize}
    \item {\bf Large acceptance}
    \item {\bf Radiation hard}
      \newline
    \item \textcolor{blue}{\bf Silicon and TRT tracker in 2T magnetic field}\\
      Measure position and momentum of charged particles
    \item \textcolor{red}{\bf Liquid argon electromagnetic calorimeter (LAr)} \\
      Measure energy of electrons and photons.
    \item {\bf Scintillating tiles hadronic calorimeter} \\
      Measure energy of jets
    \item {\bf Muon chambers}
    \end{itemize}
  \end{minipage}\\
\end{frame}

%%==========================
\begin{frame}{ Electromagnetic calorimeter (LAr) }
  \begin{minipage}{0.4\linewidth}
    \includegraphics[width=\linewidth]{MarcHDR_2f30.pdf}\newline
    \centering
    \includegraphics[width=\linewidth]{CERN-THESIS-2014-122_4f9.pdf}
  \end{minipage}
  \hfill
  \begin{minipage}{0.59\linewidth}
    \begin{itemize}
    \item $1.4$m $<r<2$m
    \item Sampling calorimeter : \\
      - absorber : lead\\
      - active material : \textcolor{blue}{\bf Liquid Argon }($88$K)\\
    \item {\bf Accordion geometry} gives uniformity and hermeticity along $\phi$.
    \item {\bf Longitudinally segmented} for pion discrimination
      
    \end{itemize}
    \centering
    \includegraphics[width=0.6\linewidth]{shower.jpg}
  \end{minipage}
\end{frame}

%=============================
%%==========================
\begin{frame}{Energy measurement in LAr}
\begin{minipage}{0.3\linewidth}
    \includegraphics[width=\linewidth]{ATLASExperiment_5f30.pdf}
\end{minipage}
\hfill
\begin{minipage}{0.69\linewidth}
\begin{itemize}
\item {\bf Signal drift time }($\sim 450$ns) {\bf too long} for collisions every $25$ns (pile-up).
\item Analog signal pass through an \textcolor{blue}{\bf bipolar filter } to reduce signal time.
Shape optimize signal over pileup and electronic noise.
\item ADC sampling every $25$ns (4 points are kept).
\item Energy computed using \textcolor{blue}{\bf calibration constants and optimal filtering of the samples}.
\end{itemize}
\end{minipage}
\begin{center}
    \includegraphics[width=\linewidth]{MarcHDR_2e3.pdf}
\end{center}

\end{frame}

%===================
\begin{frame}{Reconstruction \& Identification}
  \begin{center}
    Reconstruction links the energy deposit in detector cells to a \\ \textcolor{blue}{\bf physical particle and its properties}.
%    \vfill
    \begin{itemize}
%    \item Divide the central part ($|\eta|=|ln(tan(\theta /2) )|<2.47$) into towers of size $\Delta\eta\times\Delta\phi =0.25\times 0.25$
      %   \item Sum energies from all cells and all layers of the tower
      \item Sum energy from all layers in towers of $\Delta\eta\times\Delta\phi =0.25\times 0.25$
      \item Sliding window ($3 \times 5$ towers ) algorithm look for $2.5$~GeV of transverse energy
        \vfill
    \item {\bf Track matching and clustering} :\\
      \begin{minipage}{0.49\linewidth}
      \begin{itemize}
      \item no track $\rightarrow$ photon 
      \item track $\rightarrow$ electron
      \item conversion vertex $\rightarrow$ converted photon
      \end{itemize}
      \end{minipage}
      \hfill
      \begin{minipage}{0.49\linewidth}
        \begin{itemize}
        \item $3\times 7$ cluster  in barrel
        \item $3\times 5$ cluster  in end-cap
        \end{itemize}
      \end{minipage}
    \end{itemize}
    %% \vfill
    %% Identification is to separate prompt electrons from both jets and other electrons from either hadron decay or photon conversion.\\
    %% \textcolor{blue}{\bf A multivariate likelihood method using  23 variables} \\of energy deposit and tracking is used.
  \end{center}
\end{frame}

%=========================================

