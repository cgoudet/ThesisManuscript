\section{The Standard Model of elementary particles}
%%=============================
\begin{frame}{Particle content and interactions}
    Over the $20^{th}$ century, elementary particles have been organised into a well structured model.

  \begin{center} \includegraphics[width=\linewidth]{OPEN-PHO-CHART-2015-001.png} \end{center}
\end{frame}
%% %%=============================
\begin{frame}{Spontaneous Symmetry Breaking (SSB)}
  SSB describes a system for which its ground state has less symmetry than its Lagrangian.

  \includegraphics[width=\linewidth]{SymBreakPen.jpg}
  \begin{itemize}
  \item Unstable equilibrium has cylindrical symmetry
  \item Ground state (fallen pen) ``has chosen'' a direction.
    The cylindrical symmetry has been broken.
  \end{itemize}
\end{frame}
%%=============================
\begin{frame}{SSB in field theory}
  SSB is created by imposing a ``mexican hat'' potential on a field.
  \begin{equation}
    \label{eq:orgdea495d}
    V(\phi) = \mu^2\phi^*\phi+\lambda(\phi^*\phi)^2
  \end{equation}
  with  $\lambda>0$ and $\mu^2 <0$.
  
  \begin{minipage}{0.49\linewidth}
  \begin{itemize}
  \item Potential has rotational symmetry
  \item Ground state $|\Phi|=\sqrt{-\frac{\mu^2}{2\lambda}}= \frac{v}{\sqrt{2}}$ ($v$ = vev) breaks symmetry.
  \item Describe a massless and a massive ($m^2=v^2\lambda$) bosons.
  \end{itemize}
  \end{minipage}
  \hfill
  \begin{minipage}{0.49\linewidth}
    \includegraphics[width=\linewidth]{MexicanHatPot.jpg}
  \end{minipage}
\end{frame}
%%=============================
\begin{frame}{The Standard Model}
  
  The SM is composed of :
  
  \begin{itemize}
  \item Local gauge symmetries
    \begin{itemize}
    \item $SU(3)_c$ for strong interaction. 8 gluons couple to quarks and themselves.
    \item $SU(2)_L\times U(1)_Y$ for electroweak sector. Bosons $W^\pm$, $Z$ and photon couple to quarks, leptons, and amongst them.
    \end{itemize}
    \vfill
  \item SSB of $SU(2)_L\times U(1)_Y$ by introduction of scalar field $\Phi$
    \begin{itemize}
    \item gives mass to $W^\pm$ and $Z$.
    \item A physical and massive degree of freedom : the (Brout-Englert)-Higgs boson $H$.
    \item Yukawa coupling gives mass to fermions.
    \end{itemize}
  \end{itemize}
\end{frame}
%%=============================
\begin{frame}{Higgs boson production}
  $H$ boson predictions are function of its mass.
  
  \begin{minipage}{0.44\linewidth}
    \centering
    \begin{tikzpicture}
      \node[anchor=south west] {\includegraphics[width=0.9\linewidth]{CERN-THESIS-2014-122_1f10.pdf}};
%      \draw[step=1.0,black,thin] (0,0) grid (5,3);
      \draw[blue, line width=0.5mm, rounded corners =2pt] ( 0, 2 ) rectangle (2.5, 3.5 ) ;
      \draw[red, line width=0.5mm, rounded corners =2pt] ( 3, 1.8 ) rectangle (5, 3.5 ) ;
      \draw[green, line width=0.5mm, rounded corners =2pt] ( 0, 0.2 ) rectangle (2.5, 1.9 ) ;
      \draw[purple, line width=0.5mm, rounded corners =2pt] ( 3, 0 ) rectangle (5, 1.7 ) ;
    \end{tikzpicture}
    \\
    \includegraphics[width=0.85\linewidth]{LHCHXSWG_XS14TeV.pdf}
  \end{minipage}
  \hfill
  \begin{minipage}{0.55\linewidth}
    4 dominant production modes
    \begin{itemize}
    \item \textcolor{blue}{Gluon fusion ($ggH\simeq 86\%$) probes coupling to gluons through loop.}
    \item \textcolor{red}{Vector Boson Fusion probes direct coupling to weak bosons.}
    \item \textcolor{green}{Higgsstrahlung also probes $W^\pm$ and $Z$ couplings.}
    \item \textcolor{purple}{Associated top production probes couplings to heaviest quark.}
    \end{itemize}
  \end{minipage}
\end{frame}

%%=============================
\begin{frame}{Higgs boson decays}
  A Higgs boson with a mass around $125$~GeV opens a wide range of decay channels.
  
  \begin{minipage}{0.49\linewidth}
    \begin{tikzpicture}
      \node[anchor=south west] {\includegraphics[width=\linewidth]{Higgs_BR_LM.pdf}};
%      \draw[step=1.0,black,thin] (0,0) grid (5,5);
      \draw[red, line width=0.5mm,] ( 2.65, 0.8 ) rectangle (2.65, 5.63 ) ;

    \end{tikzpicture}
%    \includegraphics[width=\linewidth]{higgs_br.pdf}
  \end{minipage}
  \hfill
  \begin{minipage}{0.49\linewidth}
    \begin{itemize}
    \item $H\rightarrow b\bar{b}$ ($58$ \%) probes couplings to $b$ quark.
      Difficult due to large hadronic background.
    \item $H\rightarrow \tau\tau$ probes couplings to heaviest lepton.
    \item $H\rightarrow VV$ ($V=W,Z$) probes H boson couplings to weak bosons.
      Clean signature in leptonic decays of $V$ but low statistics.
    \end{itemize}
  \end{minipage}
  \begin{itemize}
    \item \textcolor{red}{$H\rightarrow\gamma\gamma$ probes H boson couplings to photon through loop.\\
      Large but smooth background. Good energy resolution.}\\

\end{itemize}
\end{frame}
%%=============================
\begin{frame}{H boson Status}
  Run 1 of the LHC (2011/2012) allowed the observation of a Higgs like particle and its properties have been measured combining ATLAS+CMS.
  \vfill
  \begin{minipage}[t]{0.59\linewidth}
    \centering
    mass measurement \\  \href{http://journals.aps.org/prl/pdf/10.1103/PhysRevLett.114. 191803}{Phys.Rev.Lett.114,191803 (2015)}\\
    \textcolor{red}{$m_H = 125.09 \pm 0.21~\text{(stat.)} \pm 0.11~\text{(syst.)} $~GeV}  
    \includegraphics[width=0.7\linewidth]{yyMass.pdf}
  \end{minipage}
  \hfill
  \begin{minipage}[t]{0.4\linewidth}
    \centering
    couplings $\kappa_i =\frac{g_{Hii}^{exp}}{g_{Hii}^{SM}}$ \\
    \href{https://cds.cern.ch/record/2158863}{CERN-EP-2016-100} \\
    \vfill
    \includegraphics[width=\linewidth]{CERN-EP-2016-100_18f.pdf} 
  \end{minipage}
\textcolor{red}{\bf The measured properties are in agreement with the SM H boson.}
\end{frame}
%%=============================
\begin{frame}{Run 2 objectives}
  \begin{minipage}{0.49\linewidth}
  \begin{itemize}
  \item LHC energy and luminosity increase \\
    $\rightarrow$ {\bf 10 times more Higgs bosons are expected}
  \item With reduced statistical uncertainties \\ $\rightarrow$ \textcolor{blue}{\bf need to reduce systematic uncertainties}
  \item Theory uncertainty reduced with ggH N$^3$LO calculation
  \item Resolution uncertainty dominant at Run 1 for couplings \\
    $\rightarrow$ {\bf Need to improve calibration resolution uncertainty}
  \end{itemize}
  \end{minipage}
  \hfill
  \begin{minipage}{0.49\linewidth}
    \centering
        \begin{tikzpicture}
      \node[anchor=south west] {\includegraphics[width=\linewidth]{crossSectionRatio-13-8TeV.pdf}};
%      \draw[step=1.0,black,thin] (0,0) grid (5,3);
      \draw[red, line width=0.5mm, rounded corners =2pt] ( 0.7, 2 ) rectangle (2.5, 2.6 ) ;
    \end{tikzpicture}
    \begin{tikzpicture}
      \node[anchor=south west] { \includegraphics[width=0.8\linewidth]{1408_7084_19t.pdf} };
%      \draw[step=1.0,black,thin] (0,0) grid (5,3);
      \draw[red, line width=0.5mm, rounded corners =2pt] ( 0, 0.95 ) rectangle (5, 1.3 ) ;
      \node[blue, draw] at (3.3,3.7) {Run 1};
      \draw[green, line width=0.5mm, rounded corners =2pt] ( 3.9, 3.4 ) rectangle (4.7, 4 ) ;
    \end{tikzpicture}
  \end{minipage}
\end{frame}
