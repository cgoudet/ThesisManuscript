\section{The Standard Model of matter}
%%=============================
\begin{frame}{Particle content of matter}
    Over the XX$^{th}$ century, elementary particles have been organised into a well structured model.

  \begin{center} \includegraphics[width=\linewidth]{OPEN-PHO-CHART-2015-001.png} \end{center}
\end{frame}
%% %%=============================
%% \begin{frame}{A mathematical framework}
%%   Matter knowledge is embedded into a well defined mathematical framework based on a Lagrangian $L$.

%%   \begin{equation}
%%     L = \frac{m \vec{\dot{q}}^2}{2} - V(\vec{q})
%%   \end{equation}

%%   The dirac lagrangian describes a massive fermion field :
%%   \begin{equation}
%%     L = \bar{\psi} ( i\slashed{\partial} - m ) \psi
%%   \end{equation}
  
%%   Imposing least action principle (similar to classical mechanic) lead to equations of motion :
%%   \begin{equation}
%%     \frac{\partial L}{\partial q} - \frac{d}{dt}\frac{\partial L}{\partial \dot{q}_i} = 0
%%   \end{equation}
  
%% \end{frame}
%% %%=============================
%% \begin{frame}{Gauge invariance}
%%   Symmetries are transformations which \textcolor{blue}{leave a system unchanged.}
%%   Imposing symmetries on a Lagrangian changes the theory it describes.
%%   \vfill
  
%%     \begin{equation}
%%       \psi(x)\rightarrow e^{i\alpha(x)}\psi(x)
%%     \end{equation}

%%     Derivative affects $e^{i\alpha}$\\
%%     $\rightarrow$ \textcolor{red}{\bf Invariance achieved by adding a field $A_\mu$ and changing $L$.} \\
%%     $\rightarrow$ mass term for $A$ ($m^2A_\mu A^\mu$) is forbidden by symmetries
%%     \begin{equation}
%%       \label{eq:orgc79752f}
%%       \partial_\mu\rightarrow D_\mu=\partial_\mu+ieA_\mu(x)
%%     \end{equation}
%%     \begin{equation}
%%       \label{eq:orge69917a}
%%       A_{\mu} \rightarrow A_\mu(x) - \frac{1}{e} {\partial_{\mu} \alpha(x)}
%%     \end{equation}


%%     \includegraphics[width=0.49\linewidth]{Iliopoulos_1f.pdf}
%%     \includegraphics[width=0.49\linewidth]{Iliopoulos_2f.pdf}


%% \end{frame}
%%=============================
\begin{frame}{Spontaneous Symmetry Breaking}
  SSB describes a system for which its ground state has less symmetry than its Lagrangian.

  \includegraphics[width=\linewidth]{SymBreakPen.jpg}
  \begin{itemize}
  \item Unstable equilibrium has cylindrical symmetry
  \item Ground state (fallen pen) ``has chosen'' a direction.
    The cylindrical symmetry has been broken.
  \end{itemize}
\end{frame}
%%=============================
\begin{frame}{SSB in field theory}
  SSB is created by imposing a ``mexican hat'' potential on a field.
  \begin{equation}
    \label{eq:orgdea495d}
    V(\phi) = \frac{1}{2}\mu^2\phi^*\phi+\frac{1}{4}\lambda(\phi^*\phi)^2
  \end{equation}
  with  $\lambda>0$ and $\mu^2 <0$.
  
  \begin{minipage}{0.49\linewidth}
  \begin{itemize}
  \item Potential has rotational symmetry
  \item Ground state $|\Phi|=\sqrt{-\frac{\mu^2}{2\lambda^2}}= \frac{\text{vev}}{\sqrt{2}}$ breaks symmetry.
  \item Describe a massless and a massive ($m^2=v^2\lambda$) bosons.
  \end{itemize}
  \end{minipage}
  \hfill
  \begin{minipage}{0.49\linewidth}
    \includegraphics[width=\linewidth]{MexicanHatPot.jpg}
  \end{minipage}
\end{frame}
%%=============================
\begin{frame}{The Standard Model}
  
  The SM is composed of :
  
  \begin{itemize}
  \item Local gauge symmetries
    \begin{itemize}
    \item $SU(3)_c$ for strong interaction. 8 gluons couple to quarks.
    \item $SU(2)_L\times U(1)_Y$ for electroweak sector. Bosons $W^\pm$, $Z$ and photon couple to quarks and leptons.
    \end{itemize}
    \vfill
  \item SSB of $SU(2)_L$ by introduction of scalar field $\Phi$
    \begin{itemize}
    \item gives mass to $W^\pm$ and $Z$.
    \item A physical and massive degree of freedom : the (Brout-Englert)-Higgs boson $H$.
    \item Yukawa coupling gives mass to fermions.
    \end{itemize}
  \end{itemize}
\end{frame}
%%=============================
\begin{frame}{Higgs boson production}
  $H$ boson predictions are function of its mass.
  
  \begin{minipage}{0.44\linewidth}
    \centering
    \includegraphics[width=\linewidth]{CERN-THESIS-2014-122_1f10.pdf}\\
    \includegraphics[width=\linewidth]{YRHXS_Summary_fig3.pdf}
  \end{minipage}
  \hfill
  \begin{minipage}{0.55\linewidth}
    4 dominant production modes
    \begin{itemize}
    \item Gluon fusion ($ggH\simeq 86\%$) probes coupling to gluons through loop.
    \item Vector Boson Fusion probes direct coupling to electroweak bosons.
    \item Higgsstrahlung also probes $W\pm$ and $Z$ couplings.
    \item Associated top production probes couplings to heaviest quark.
    \end{itemize}
  \end{minipage}
\end{frame}

%%=============================
\begin{frame}{Higgs boson decays}
  A Higgs boson with a mass around $125$~GeV opens a wide range of decay channels.
  
  \begin{minipage}{0.49\linewidth}
    \centering
    \includegraphics[width=\linewidth]{higgs_br.pdf}
  \end{minipage}
  \hfill
  \begin{minipage}{0.49\linewidth}
    \begin{itemize}
    \item $H\rightarrow bb$ ($57$ \%) probes couplings to $b$ quark.
      Difficult due to large hadronic background.
    \item $H\rightarrow \tau\tau$ probes couplings to heaviest lepton.
    \item $H\rightarrow VV$ ($V=W,Z$) probes H boson couplings to EW bosons.
      Clean signature in leptonic decays of $V$ but low statistics.
    \item $H\rightarrow\gamma\gamma$ probes H boson couplings to photon through loop.
      Large but smooth background. Good energy resolution.
    \end{itemize}
  \end{minipage}
  
\end{frame}
%%=============================
\begin{frame}{H boson Status}
  Run 1 of the LHC (2011/2012) allowed the observation of a Higgs like particles and its properties have been measured combining ATLAS+CMS.
  \vfill
  \begin{minipage}[t]{0.59\linewidth}
    \centering
    mass measurement \\  \href{http://journals.aps.org/prl/pdf/10.1103/PhysRevLett.114.191803}{PhysRevLett.114.191803}\\
    \textcolor{red}{$m_H = 125.09 \pm 0.21 \text{(stat)} \pm 0.11 \text{(syst)} $~GeV}  
    \includegraphics[width=0.7\linewidth]{yyMass.pdf}
  \end{minipage}
  \hfill
  \begin{minipage}[t]{0.4\linewidth}
    \centering
    couplings $\kappa_i =\frac{g_i^{exp}}{g_i^{SM}}$ \\
    \href{https://cds.cern.ch/record/2158863}{CERN-EP-2016-100} \\
    \vfill
    \includegraphics[width=\linewidth]{CERN-EP-2016-100_18f.pdf} 
  \end{minipage}
\textcolor{red}{\bf The measured properties are in agreement with the SM H boson.}
\end{frame}
%%=============================
\begin{frame}{Run 2 objectives}
  \begin{minipage}{0.49\linewidth}
  \begin{itemize}
  \item LHC energy and luminosity increase \\
    $\rightarrow$ {\bf 10 times more Higgses are expected}
  \item With reduced statistical uncertainties \\ $\rightarrow$ \textcolor{blue}{\bf need to reduce systematic uncertainties}
  \item Theory uncertainty reduced with ggH N$^3$LO calculation
  \item Resolution uncertainty dominant at run 1 for couplings \\
    $\rightarrow$ {\bf Need to improve calibration resolution uncertainty}
  \end{itemize}
  \end{minipage}
  \hfill
  \begin{minipage}{0.49\linewidth}
    \centering
    \includegraphics[width=\linewidth]{crossSectionRatio-13-8TeV.pdf}\\
    \begin{tikzpicture}
      \node[anchor=south west] { \includegraphics[width=0.8\linewidth]{1408_7084_19t.pdf} };
%      \draw[step=1.0,black,thin] (0,0) grid (5,3);
      \draw[red, line width=0.5mm, rounded corners =2pt] ( 0, 0.6 ) rectangle (5, 1.3 ) ;
    \end{tikzpicture}
  \end{minipage}
\end{frame}
