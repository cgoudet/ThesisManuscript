\begin{frame}{Identification variables}
  \begin{center}
\includegraphics[width=0.45\linewidth]{CONF-2014-032_1t.pdf}
\end{center}
\end{frame}

\begin{frame}{Calibration in-situ : run 1  results and uncertainties}
\begin{minipage}{0.64\linewidth}
  Uncertainties are evaluated as the difference between official scales and the ones measured with a changed parameter. 
  They include :
  \begin{itemize}
  \item electron identification quality from medium to tight.
  \item Z mass window 
  \item electron $p_T$ cut
  \item uncertainties on efficiencies scale factors
  \item energy loss through bremshtrahlung
  \item background
  \item pile-up
  \item measurement method
  \end{itemize}
\end{minipage}
\begin{minipage}{0.35\linewidth}
    \includegraphics[width=\linewidth]{CERN-PH-EP-2014-153_26f.pdf}\\
    \includegraphics[width=\linewidth]{CERN-PH-EP-2014-153_27f.pdf}
\end{minipage}
\end{frame}

%%========================================
\begin{frame}{Run 2 prerecommandations}
Run 2 early analyses need scales factors for 13TeV but not enough data will be available.
Need to \textcolor{blue}{ \bf estimate run 2 scales from run 1 data}.
\newline
Pre-recommandations are computed using $8$~TeV data reprocessed with :
\begin{itemize}
\item new detector geometry
\item new reconstruction algorithm
\item new calibration machine learning
\end{itemize} 
  \begin{minipage}{0.49\linewidth}
    \includegraphics[width=\linewidth]{/home/goudet/Documents/LAL/Shared/PlotsGoudet/Calibration/PreRecommandations/PreRec_alpha.pdf}
  \end{minipage}
  \begin{minipage}{0.49\linewidth}
    \includegraphics[width=\linewidth]{/home/goudet/Documents/LAL/Shared/PlotsGoudet/Calibration/PreRecommandations/PreRec_c.pdf}
  \end{minipage}
%Binning was changed from $34$ bins to $68$ due to the observation of sub-structures.
\end{frame}

%%========================================

\begin{frame}{Calibration in-situ : run 2 pre-recommandations systematics}
2012 systematics are used for the pre-recommandations. \\
{\bf Two more systematics are added in quadrature } :
\begin{itemize}
\item Increasing the number of bin for $\alpha$ shows sub-patterns. 
  Systematic is defined as difference between a bin value and the average of its sub-bins.
\item Pre-recommandations being computed with 8TeV datasets, one needs to evaluate the impact of the center of mass energy.
Systematic is defined as the scale measured from $13$~TeV MC on $8TeV$ templates.
\end{itemize}
  \begin{minipage}{0.49\linewidth}
    \includegraphics[width=\linewidth]{/home/goudet/Documents/LAL/Shared/PlotsGoudet/Calibration/PreRecommandations/PreRecSyst_alpha.pdf}
  \end{minipage}
  \hfill
  \begin{minipage}{0.49\linewidth}
    \includegraphics[width=\linewidth]{/home/goudet/Documents/LAL/Shared/PlotsGoudet/Calibration/PreRecommandations/PreRecSyst_c.pdf}
  \end{minipage}\\
\end{frame}

%======================================
\begin{frame}{ATLAS run 1 H boson mass measurement}
\centering
$$m_H = 125.36 \pm 0.37 \text{(stat)} \pm 0.18 \text{(syst)}$$
\begin{minipage}{0.49\linewidth}
  \includegraphics[width=\linewidth]{1406_3827_8f.pdf}
\end{minipage}
\hfill
\begin{minipage}{0.49\linewidth}
  \begin{tikzpicture}
    \node[anchor=south west] { \includegraphics[width=\linewidth]{1406_3827_4t.pdf} };
    \draw[step=1.0,black,thin] (0,0) grid (10,4);
    \draw[red, line width=0.5mm, rounded corners =2pt] ( 0.1, 1.05 ) rectangle ( 6.1, 1.35 ) ;
  \end{tikzpicture}
\end{minipage}
    {\bf Statistical uncertainties highly dominant.}\\
    \begin{center}   Run 2 will increase sensitivity to systematics.\end{center}
\end{frame}
