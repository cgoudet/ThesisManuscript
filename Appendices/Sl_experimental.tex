\begin{frame}{Identification variables}
  \begin{center}
\includegraphics[width=0.45\linewidth]{CONF-2014-032_1t.pdf}
\end{center}
\end{frame}

%=======================================================
\begin{frame}{Reconstruction \& Identification efficiencies}
  Not all electrons pass the reconstruction and identification criteria. \\
  {\bf 3 menus with increasing purity ( but deceasing efficiencies) are defined : loose, medium, tight}.
  The efficiency of these procedures is given as a function of the $p_T$ and $\eta$.\\
\begin{minipage}{0.49\linewidth}
  \includegraphics[width=\linewidth]{CONF-2014-032_30fa.pdf}
\end{minipage}
\begin{minipage}{0.49\linewidth}
  \includegraphics[width=\linewidth]{CONF-2014-032_30fb.pdf}
\end{minipage}
\end{frame}

%=======================================================

\begin{frame}{Calibration in-situ : run 1  results and uncertainties}
\begin{minipage}{0.64\linewidth}
  Uncertainties are evaluated as the difference between official scales and the ones measured with a changed parameter. 
  They include :
  \begin{itemize}
  \item electron identification quality from medium to tight.
  \item Z mass window 
  \item electron $p_T$ cut
  \item uncertainties on efficiencies scale factors
  \item energy loss through bremshtrahlung
  \item background
  \item pile-up
  \item measurement method
  \end{itemize}
\end{minipage}
\begin{minipage}{0.35\linewidth}
    \includegraphics[width=\linewidth]{CERN-PH-EP-2014-153_26f.pdf}\\
    \includegraphics[width=\linewidth]{CERN-PH-EP-2014-153_27f.pdf}
\end{minipage}
\end{frame}
