\chapter*{Remerciements}
\addcontentsline{toc}{chapter}{Remerciements}

Tout d'abord, je veux remercier Louis Fayard de m'avoir accompagné pendant de nombreuses années.
Merci de m'avoir fait découvrir la physique des particules pour mon stage de L3 (aidé quand même par une découverte majeure) et d'avoir suivi mon parcours pour me  permettre de commencer cette thèse au LAL.
Merci de tous tes efforts pour le financement qui n'était pas assuré.
Merci Louis pour ton encadrement soutenu en début de thèse qui m'a permis d'apprendre énormément et de toujours repousser mes certitudes tout en me permettant d'avancer vite.
Merci de m'avoir permis d'expérimenter mes propres solutions.
Ton mot d'ordre ``l'étudiant a le plus souvent tort'' m'a forcé à améliorer ma rigueur et mon argumentation.
J'ai ainsi pu me voir grandir au cours de ces trois années en voyant diminuer ma fréquence d'erreur.
J'ai énormément appris à tes côtés et je t'en suis reconnaissant.

Je remercie mon jury Glen Cowan, Claude Duhr, Fabrice Hubaut, Luigi Rolandi, Achille Stocchi et Guillaume Unal qui ont accepté de venir évaluer mon travail.
Je vous remercie également pour tous vos conseils et vos critiques constructives sur mon manuscrit.
En particulier, je remercie Fabrice et Luigi pour avoir accepté d'être rapporteurs.
Achille, j'apprécie ta volonté de t'impliquer auprès des doctorants de ton laboratoire en participant à leur jury de thèse, malgré les nombreuses obligations auxquelles tu dois faire face.


Je souhaite remercier toutes les personnes avec qui j'ai été amené à travailler au cours de ces trois années.
Merci Guillaume pour ta patience durant les nombreuses discussions que nous avons eu et au cours desquelles j'ai énormément appris.
Merci Marc pour ton aide sur tous les sujets possibles depuis mon arrivée en stage.
Au delà de la physique, j'ai encore les exercices de dactylo que tu m'avais donnés en stage pour m'apprendre à taper à dix doigts.
Merci à Daniel pour les nombreuses discussions : se faire présenter le calo par son papa est une expérience assez exceptionnelle.
Je remercie Glen pour ses passages multiples au LAL au cours desquels j'ai pu comprendre de nombreux aspects de statistiques.
Au passage, merci de m'avoir dédicacé personnellement  ton livre.
Merci à l'ensemble de l'équipe ATLAS du LAL pour son accueil et les échanges productifs tout au long de ma thèse.
Merci enfin à tout le personnel administratif du LAL et de l'ED qui s'est assuré que mon passage au LAL se passe dans les meilleures conditions.
Geneviève, merci pour ta patience lorsqu'il qui me fallait remplir les formulaires de mission et de ne pas m'avoir oublié lors des restes de viennoiseries.

Bien que n'ayant pas contribué directement à ma thèse, je souhaite remercier Olivier Bondu et David d'Enterria de m'avoir sélectionné pour le programme Summer Student 2013.
J'ai beaucoup apprécié travailler à vos côtés, autant sur le plan humain que scientifique.
Olivier, tu as changé ma vie en me faisant découvrir git.

Je souhaite remercier tous les doctorants du LAL qui ont croisé ma route tout au long de ces années et qui ont contribué à cette thèse.
Merci à Narei Lorenzo de m'avoir ouvert les portes du LAL en m'obtenant mon premier stage.
Merci à Estelle Scifo pour son encadrement et ses enseignements tout au long de ce stage, qui m'ont donné envie de continuer dans cette voie.
Merci à Cyril et Marta qui m'ont accueilli dans ce groupe des doctorants.
Enfin merci aux plus jeunes : Antinéa, Antoine, Charles et Corentin.
Les soirées films ont bien rythmé ces deux dernières années et je suis confiant quant à l'ambiance du groupe pour les prochaines années.
Antinéa, ce sera maintenant à toi de ``pusher sur PlotFunctions''.


Je souhaite remercier les membres des associations D2I2 et Synapse pour leur implication auprès des doctorants et leur impact auprès de moi.
Merci Marie-Coralie de m'avoir ouvert les yeux sur l'insertion professionnelle.
Grâce à toi j'ai pu aborder la fin de thèse avec sérénité.
Merci à Pauline, Alice, Anastasia et Clément pour le travail que nous avons mené ensemble au sein de Synapse.


Je souhaite remercier toutes les personnes qui ont accompagné tout mon parcours de physique à Orsay.
Merci à Sébastien pour sa bonne humeur communicative et ses histoires invraisemblables.
Victor, Jordan, Erwan et Kevin merci pour les nombreuses nuit que j'ai passées sur votre canapé.
Merci à Claire, Maelle, Pierre, Julia et Baptiste pour les nombreux délires en magistère comme en thèse.
Enfin merci à Steven pour ces 6 années où l'on ne s'est jamais vraiment quittés.
Merci pour les TP de L3 (on avait clairement une meilleure note quand tu rédigeais), le summer student (les saucisses aux herbes pour ne mentionner qu'une des nombreuses anecdotes) et bien sûr les trois dernières années : les films (plus ou moins) pourris aux soirées, tes arrivées impromptues dans mon bureau quand tu t'ennuyais et de temps en temps une discussion sérieuse de stat.

Pour finir je veux remercier ma famille, présents ou disparus,  qui m'ont soutenu tout au long de ces nombreuses années d'études.
Enfin, merci à toute la famille qui a contribué à s'assurer que mon placard restait plein de spécialités du sud : le foie gras du parrain et les pâtés de mamie.
Finalement je veux remercier mes parents pour avoir toujours fait en sorte que je puisse étudier dans les meilleures conditions.
Ma réussite est surtout la vôtre.



