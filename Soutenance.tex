\documentclass[a4paper]{beamer}
\usepackage[english]{babel}
\usepackage[utf8]{inputenc}
\usepackage{graphicx}
\usepackage{subcaption}
\usepackage{caption}
\usepackage{appendixnumberbeamer}
\usepackage{tikz}
\usepackage{csvsimple}
\usepackage{tabularx}
\usepackage{adjustbox}
\usepackage{pdfpages}
\usepackage{cleveref}
%\usepackage[orientation=paysage,size=A4]{beamerposter}
\usepackage{slashed}
\captionsetup{labelformat=empty,labelsep=none}
\beamertemplatenavigationsymbolsempty
\usetheme{Boadilla}

\title[Energy calibration \& Higgs couplings]{Calibration of the ATLAS electromagnetic calorimeter and measurement of the couplings of the (Brout-Englert-)Higgs boson in the diphoton channel}
\author[Goudet]{Christophe Goudet}
\institute[LAL]{\includegraphics[width=0.3\linewidth]{LAL.jpg} }

\date[Orsay, \today]{PhD defense \\ Orsay, \today}
\graphicspath{{/home/goudet/Documents/LAL/ExternalPlot/}{/home/goudet/Documents/LAL/Zim/Calibration/TemplateMethod/PlotsIllustration/}}

\begin{document}
\transboxin
\begin{frame}
\maketitle
\end{frame}

\begin{frame}{Introduction}
\tableofcontents
\end{frame}

\section{The Standard Model of matter}
%%=============================
\begin{frame}{Particle content of matter}
    Over the XX$^{th}$ century, elementary particles have been organised into a well structured model.

  \begin{center} \includegraphics[width=\linewidth]{OPEN-PHO-CHART-2015-001.png} \end{center}
\end{frame}
%%=============================
\begin{frame}{A mathematical framework}
  Matter knowledge is embedded into a well defined mathematical framework based on a Lagrangian $L$.

  \begin{equation}
    L = \frac{m \vec{\dot{q}}^2}{2} - V(\vec{q})
  \end{equation}

  The dirac lagrangian describes a massive fermion field :
  \begin{equation}
    L = \bar{\psi} ( i\slashed{\partial} - m ) \psi
  \end{equation}
  
  Imposing least action principle (similar to classical mechanic) lead to equations of motion :
  \begin{equation}
    \frac{\partial L}{\partial q} - \frac{d}{dt}\frac{\partial L}{\partial \dot{q}_i} = 0
  \end{equation}
  
\end{frame}
%%=============================
\begin{frame}{Gauge invariance}
  Symmetries are transformations which \textcolor{blue}{leave a system unchanged.}
  Imposing symmetries on a Lagrangian changes the theory it describes.
  \vfill
  
    \begin{equation}
      \psi(x)\rightarrow e^{i\alpha(x)}\psi(x)
    \end{equation}

    Derivative affects $e^{i\alpha}$\\
    $\rightarrow$ \textcolor{red}{\bf Invariance achieved by adding a field $A_\mu$ and changing $L$.}
    \begin{equation}
      \label{eq:orgc79752f}
      \partial_\mu\rightarrow D_\mu=\partial_\mu+ieA_\mu(x)
    \end{equation}
    \begin{equation}
      \label{eq:orge69917a}
      A_{\mu} \rightarrow A_\mu(x) - \frac{1}{e} {\partial_{\mu} \alpha(x)}
    \end{equation}


    \includegraphics[width=0.49\linewidth]{Iliopoulos_1f.pdf}
    \includegraphics[width=0.49\linewidth]{Iliopoulos_2f.pdf}


\end{frame}
%%=============================
\begin{frame}{Spontaneous symmetry breaking}
\end{frame}
%%=============================
\begin{frame}{The Standard Model}
  list of properties
  plot de particle content
\end{frame}
%%=============================
\begin{frame}{Higgs measurement}
\end{frame}
%%=============================
\begin{frame}{Run 2 objectives}
\end{frame}

%%=============================
%% \begin{frame}{Introduction}
%%   \begin{minipage}{0.6\linewidth}
%%     \begin{itemize}
%%     \item {\bf Higgs precision measurement led by electrons and photons} (and muons) .\\
%%       \begin{center} \textcolor{red}{$m_H = 125.09 \pm 0.21 \text{(stat)} \pm 0.11 \text{(syst)} $~GeV} 
%% (ATLAS+CMS) \href{http://journals.aps.org/prl/pdf/10.1103/PhysRevLett.114.191803}{PhysRevLett.114.191803}\end{center}

%%     \item Run 2 ongoing at an increased center of mass energy of $13$~TeV. {\bf 30 times more Higgses are expected}
%%     \item Higgs couplings may bring hints of new physics : \\ $$\mu = \frac{\sigma^{meas}}{\sigma^{SM}}$$
%%     \item With reduced statistical uncertainties \\ $\rightarrow$ \textcolor{blue}{\bf need to reduce systematic uncertainties}.
%%     \item Calibration is a important source of systematic.
%%       Needs to be improved in Run 2.
%%     \end{itemize}
%%   \end{minipage}
%%   \hfill
%%   \begin{minipage}{0.33\linewidth}
%%     \includegraphics[width=\linewidth]{yyMass.pdf}\\
%%     \includegraphics[width=\linewidth]{1408_7084_19f.pdf}\\
%%   \end{minipage}
%% \end{frame}


\section{Experimental conditions and data processing}

\begin{frame}{The LHC}
\end{frame}
%%===========================
\begin{frame}{General purpose apparatus}
\end{frame}
%%===========================
\begin{frame}{ATLAS experiment}
  \begin{minipage}{0.49\linewidth}
    \includegraphics[width=\linewidth]{ATLASExperiment_1f1.pdf}\\
    Performance goals of the ATLAS detector\\
    \includegraphics[width=\linewidth]{ATLASExperiment_1t1.pdf}
  \end{minipage}
  \hfill
  \begin{minipage}{0.49\linewidth}
    \begin{itemize}
    \item {\bf Large acceptance}
    \item {\bf Radiation hard}
      \newline
    \item \textcolor{blue}{\bf Silicon and TRT tracker in 2T magnetic field}\\
      Measure position and momentum of charged particles
    \item \textcolor{blue}{\bf Liquid argon electromagnetic calorimeter (LAr)} \\
      Measure energy of electrons and photons.
    \item {\bf Scintillating tiles hadronic calorimeter} \\
      Measure energy of jets
    \item {\bf Muon chambers}
    \end{itemize}
  \end{minipage}\\
\end{frame}

%%==========================
\begin{frame}{ Electromagnetic calorimeter (LAr) }
  \begin{minipage}{0.4\linewidth}
    \includegraphics[width=\linewidth]{MarcHDR_2f30.pdf}\newline
    \centering
    \includegraphics[width=\linewidth]{ATLASExperiment_5f4.pdf}
  \end{minipage}
  \hfill
  \begin{minipage}{0.59\linewidth}
    \begin{itemize}
    \item $1.4$m $<r<2$m
    \item Sampling calorimeter : \\
      - absorber : lead\\
      - active material : \textcolor{blue}{\bf Liquid Argon }($88$K)\\
    \item {\bf Accordion geometry} gives uniformity and hermeticity along $\phi$.
    \item {\bf Longitudinally segmented} for pion discrimination
      
    \end{itemize}
    \centering
    \includegraphics[width=0.6\linewidth]{shower.jpg}
  \end{minipage}
\end{frame}

%=============================
\begin{frame}{Data recording}
  speak of OFC to transform electric signa to 
  \end{frame}
%=============================
\begin{frame}{EM object reconstruction}
  track reconstruction and EM clusters
\end{frame}
%%=============================
\begin{frame}{EM objects calibration}
\end{frame}
%%=============================
\begin{frame}{Energy scale factors}
  \begin{minipage}{0.49\linewidth}
    After MVA calibration, mass distribution of $Z\rightarrow ee$ for data and MC still have \\{\bf discrepancy}.
    \newline
    \textcolor{blue}{\bf A data-driven analysis } is performed to match data to MC distribution \\(relative matching).
  \end{minipage}
  \hfill
  \begin{minipage}{0.49\linewidth}
    \includegraphics[width=\linewidth]{CalibSupNote_Distri_m12_uncorrected.pdf}
  \end{minipage}

A correction, applied to both electrons of Z decay, is computed to shift the central value of data distribution : 
\begin{center} \textcolor{blue}{\bf energy scale factor ($\alpha$)} \end{center}
$$E^{corr}=E^{meas}(1+\alpha)$$
\end{frame}

%%============================
\begin{frame}{Resolution constant term}
  \input{/home/goudet/Documents/LAL/ExternalPlot/ResolutionConstantTerm.tex}
  \begin{itemize}
  \item $a$ : sampling term ( $10\%$). Linked to the fluctuations of electromagnetic showers. \\Can be simulated.
  \item $b/E$ : noise term ( $350cosh(\eta )$~MeV ). Measured in dedicated runs.
  \item {\bf c : constant term ($0.7\%$)}. Must be measured on data.
  \end{itemize}
  We observe that data distribution is larger than MC. 
  An {\bf additional constant term (C)} is measure to enlarge MC up to the data width.
  Both MC electrons undergo the correction :
  \begin{center}\textcolor{blue}{\bf Resolution constant term (C) }\end{center}
  $$E^{corr} = E^{meas}(1+N(0,1)*C)$$
  $N(0,1)$ : a Gaussian distributed random number
\end{frame}

%%===================================
\begin{frame}{Template method}
\begin{minipage}{0.59\linewidth}
  The template method is used to measure $\alpha$ and $C$ simultaneously.
\begin{itemize}
\item Create distorded MC (templates) with test values of $\alpha$ and $C$.
\item \textcolor{blue}{\bf Compute $\chi^2$ between Z mass distribution of data and template}.
\item \textcolor{blue}{\bf Fit the minimum of the $\chi^2$ distribution} in the ($\alpha,C$) plane.
\item Fit performed in 2 steps of 1D fits : 
\begin{itemize}
\item fit $\chi^2=f(\alpha)$ at constant $C$ (lines) $\rightarrow (\alpha_{min}, \chi^2_{min})$ .
\item fit $\chi^2_{min}=f(C)\rightarrow (C, \Delta C)$
\item project $C$ in $\alpha_{min}=f(C)$, corresponding bin gives $(\alpha, \Delta\alpha)$.
\end{itemize}
\end{itemize}
  \includegraphics[width=0.325\linewidth]{MC6_0_0_chi2FitNonConstVar_10.pdf}
  \includegraphics[width=0.325\linewidth]{MC6_0_0_chi2FitConstVar.pdf}
  \includegraphics[width=0.325\linewidth]{MC6_0_0_corAngle.pdf}
\end{minipage}
\hfill
\begin{minipage}{0.4\linewidth}
  \includegraphics[width=\linewidth]{MC6_0_0_CompareAlpha.pdf}\\
  \includegraphics[width=\linewidth]{MC6_0_0_chiMatrix.pdf}\\
\end{minipage}
\end{frame}



%%===================================
\begin{frame}{Run 2 results}
  Scales are measured with 13TeV data at 25ns
  % \begin{itemize}
  % \item Data are corrected with energy scales from pre-recommandations
  % \item MC is {\bf not} smeared with pre-rec
  % \end{itemize}
  \begin{minipage}{0.49\linewidth} 
    \includegraphics[width=\linewidth]{CalibSupNote_CS_alphaOff.pdf}
  \end{minipage}
  \hfill
  \begin{minipage}{0.49\linewidth}
    \includegraphics[width=\linewidth]{CalibSupNote_CS_cOff.pdf}
  \end{minipage}
  {\bf $\alpha$  discrepancies are below $0.1\%$ } out of the crack ($1.37<|\eta|<1.55$).
\end{frame}
%===============================================
\begin{frame}{Uncertainties}
\end{frame}
%===============================================
\begin{frame}{Runs comparison}
\end{frame}
%===============================================
\section{Measurement of Higgs boson couplings}
\begin{frame}{Higgs boson at the LHC}
  \begin{minipage}{0.6\linewidth}
  \begin{itemize}
  \item Higgs boson predicted in 1964, discovered in 2012.
  \item Gives mass to weak boson, and fermions through Yukawa coupling.
  \item {\bf Several production mode are available at the LHC.}
    \begin{itemize}
      \item ggH : $gg\rightarrow H$
      \item VBF : $qq\rightarrow Hjj$ 
      \item VH : $Z(W)\rightarrow Z(W)H$
      \item ttH : $t\bar{t}\rightarrow t\bar{t}H$
    \end{itemize}
  \item At a mass of $125$~GeV, many decay modes available : 
    \begin{itemize}
    \item $H\rightarrow b\bar{b}$ : dominant decay mode ( $\sim 57\%$ ) but high background in hadronic machines.
    \item $H\rightarrow 4l$ : low expected events, almost no background.
    \item \textcolor{blue}{ $H\rightarrow\gamma\gamma$ : low branching ratio ($0.28\%$) but clean signature. High but smooth background.}
    \end{itemize}
  \end{itemize}
\end{minipage}
  \begin{minipage}{0.33\linewidth}
    \includegraphics[width=\linewidth]{Higgs_XS_8TeV_sq.pdf}\\
    \includegraphics[width=\linewidth]{higgs_br.pdf}\\
\end{minipage}
\end{frame}


\begin{frame}{Likelihood Method}
A function ({\bf likelihood}) is built to {\bf evaluate the best set of parameters ($\vec{\mu}$,$\vec{\theta}$)} of a model to agree the best with a dataset in a category.

$$\mathcal{L}=\underbrace{\frac{(n_{s}(\vec{\mu},\vec{\theta})+b)^{n_{obs}}}{n_{obs}!} e^{-(n_{s}(\vec{\mu},\vec{\theta})+b)}}_{\textcolor{red}{\text{(1)}}}  \overbrace{\prod_j^{n_{obs}}\psi(\vec{x_j};\vec{\mu},\vec{\theta})}^{\textcolor{violet}{\text{(2)}}} \underbrace{e^{-\frac{\theta^2}{2}}}_{\textcolor{blue}{\text{(3)}}}$$
\vfill
\begin{minipage}{0.49\linewidth}
\textcolor{red}{(1) {\bf Poissonian law} to evaluate the probability to observe $n_{obs}$($\equiv$ signal $+$ background) events when $(n_s+b)$ are expected.}\newline
\textcolor{violet}{(2) {\bf Probability density function} of the observables $\vec{x}$ (diphoton invariant mass for example) for the $j^{th}$ event.}\newline
\textcolor{blue}{(3) Constraint on the nuisance parameter $\theta$. See next slide.}\newline
\end{minipage}
\begin{minipage}{0.49\linewidth}
\includegraphics[width=\linewidth]{Cgam_009.png}
\end{minipage}
\end{frame}




\begin{frame}{Nuisance parameters}
There are some {\bf external measurements}  that contribute to the likelihood and have some {\bf uncertainties}. 
A {\bf free nuisance parameter} is added for each of these measurements.
In order  to take into account these external measurements, a {\bf constraint is put on these nuisance parameters}. 
\vfill
For example, the luminosity is re-defined as  $L(1+\delta_L \theta_L)$, with $\theta_L$ the nuisance parameter and $\delta_L$ the uncertainty on the luminosity (assumed to be Gaussian).
In this case, a Gaussian constraint is chosen.

The contribution from luminosity will hence be :
$$L(1+\delta_L \theta_L)e^{-\theta_L^2/2}$$
\vfill
\textcolor{blue}{\bf Error Estimation}\newline
A test statistic is defined as : $t_\mu=-2ln\frac{\mathcal{L}(\mu,\hat{\hat{\theta}})}{\mathcal{L}(\hat{\mu},\hat{\theta})}$, with $\hat{\theta}$ and $\hat{\mu}$ the best (fitted) parameters, and $\hat{\hat{\theta}}$ the fitted nuisance parameters for a fixed $\mu$.\newline
Uncertainty are given by : \textcolor{red}{$\mathbf{t_{\hat{\mu}\pm 1\sigma}=1}$} and \textcolor{red}{$\mathbf{t_{\hat{\mu}\pm 2\sigma}=4}$} in 1D Gaussian limit.
\end{frame}


\begin{frame}
\maketitle
\end{frame}
%% ###################################################################################
%%###################################################################################
%%###################################################################################
\appendix
\begin{frame}{Identification variables}
  \begin{center}
\includegraphics[width=0.45\linewidth]{CONF-2014-032_1t.pdf}
\end{center}
\end{frame}

\begin{frame}{Calibration in-situ : run 1  results and uncertainties}
\begin{minipage}{0.64\linewidth}
  Uncertainties are evaluated as the difference between official scales and the ones measured with a changed parameter. 
  They include :
  \begin{itemize}
  \item electron identification quality from medium to tight.
  \item Z mass window 
  \item electron $p_T$ cut
  \item uncertainties on efficiencies scale factors
  \item energy loss through bremshtrahlung
  \item background
  \item pile-up
  \item measurement method
  \end{itemize}
\end{minipage}
\begin{minipage}{0.35\linewidth}
    \includegraphics[width=\linewidth]{CERN-PH-EP-2014-153_26f.pdf}\\
    \includegraphics[width=\linewidth]{CERN-PH-EP-2014-153_27f.pdf}
\end{minipage}
\end{frame}

%%========================================
\begin{frame}{Run 2 prerecommandations}
Run 2 early analyses need scales factors for 13TeV but not enough data will be available.
Need to \textcolor{blue}{ \bf estimate run 2 scales from run 1 data}.
\newline
Pre-recommandations are computed using $8$~TeV data reprocessed with :
\begin{itemize}
\item new detector geometry
\item new reconstruction algorithm
\item new calibration machine learning
\end{itemize} 
  \begin{minipage}{0.49\linewidth}
    \includegraphics[width=\linewidth]{/home/goudet/Documents/LAL/Shared/PlotsGoudet/Calibration/PreRecommandations/PreRec_alpha.pdf}
  \end{minipage}
  \begin{minipage}{0.49\linewidth}
    \includegraphics[width=\linewidth]{/home/goudet/Documents/LAL/Shared/PlotsGoudet/Calibration/PreRecommandations/PreRec_c.pdf}
  \end{minipage}
%Binning was changed from $34$ bins to $68$ due to the observation of sub-structures.
\end{frame}

%%========================================

\begin{frame}{Calibration in-situ : run 2 pre-recommandations systematics}
2012 systematics are used for the pre-recommandations. \\
{\bf Two more systematics are added in quadrature } :
\begin{itemize}
\item Increasing the number of bin for $\alpha$ shows sub-patterns. 
  Systematic is defined as difference between a bin value and the average of its sub-bins.
\item Pre-recommandations being computed with 8TeV datasets, one needs to evaluate the impact of the center of mass energy.
Systematic is defined as the scale measured from $13$~TeV MC on $8TeV$ templates.
\end{itemize}
  \begin{minipage}{0.49\linewidth}
    \includegraphics[width=\linewidth]{/home/goudet/Documents/LAL/Shared/PlotsGoudet/Calibration/PreRecommandations/PreRecSyst_alpha.pdf}
  \end{minipage}
  \hfill
  \begin{minipage}{0.49\linewidth}
    \includegraphics[width=\linewidth]{/home/goudet/Documents/LAL/Shared/PlotsGoudet/Calibration/PreRecommandations/PreRecSyst_c.pdf}
  \end{minipage}\\
\end{frame}

%======================================
\begin{frame}{ATLAS run 1 H boson mass measurement}
\centering
$$m_H = 125.36 \pm 0.37 \text{(stat)} \pm 0.18 \text{(syst)}$$
\begin{minipage}{0.49\linewidth}
  \includegraphics[width=\linewidth]{1406_3827_8f.pdf}
\end{minipage}
\hfill
\begin{minipage}{0.49\linewidth}
  \begin{tikzpicture}
    \node[anchor=south west] { \includegraphics[width=\linewidth]{1406_3827_4t.pdf} };
    \draw[step=1.0,black,thin] (0,0) grid (10,4);
    \draw[red, line width=0.5mm, rounded corners =2pt] ( 0.1, 1.05 ) rectangle ( 6.1, 1.35 ) ;
  \end{tikzpicture}
\end{minipage}
    {\bf Statistical uncertainties highly dominant.}\\
    \begin{center}   Run 2 will increase sensitivity to systematics.\end{center}
\end{frame}

\end{document}

